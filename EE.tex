\documentclass[12pt]{article}
\usepackage{color}
\usepackage{parskip}
\usepackage{amsmath}
\usepackage{amssymb}
\usepackage{graphicx}
\usepackage{listings}
\usepackage{setspace}
\usepackage{geometry}
\usepackage{dirtytalk}
\usepackage{indentfirst}
\usepackage{anyfontsize}
\usepackage[utf8]{inputenc}

\usepackage{natbib}

\geometry{letterpaper, portrait, margin=1in}
\parindent=0.5in

\definecolor{codegray}{rgb}{0.2,0.2,0.2}
\definecolor{codepurple}{rgb}{0.58,0,0.82}
\definecolor{backcolor}{rgb}{0.95,0.95,0.92}

\lstdefinestyle{scheme}
  {backgroundcolor=\color{backcolor},   
  commentstyle=\color{blue},
  keywordstyle=\color{magenta},
  numberstyle=\tiny\color{codegray},
  stringstyle=\color{codepurple},
  basicstyle=\footnotesize\ttfamily,
  morekeywords={*},
  breakatwhitespace=false,         
  breaklines=true,                 
  captionpos=t,                    
  keepspaces=true,                 
  numbers=left,                    
  numbersep=5pt,                  
  showspaces=false,                
  showstringspaces=false,
  showtabs=false,                  
  tabsize=2,
  title=\lstname,
  language=Python
}

\lstset{style=scheme}

\doublespace{}
\title{The use of reaction diffusion equations for the simulation of physical phenomena}
\author{Simon Abrelat}
\date{\vspace{-5ex}}

\DeclareUnicodeCharacter{2212}{-}
\begin{document}
\large
{\fontsize{12}{14.4}
  {\singlespace{}
  \pagenumbering{gobble}
  \maketitle
  \begin{center}
  \vspace{4mm}
  002129--0004 \\
  \vspace{4mm}
  Math HL IA \\
  \vspace{4mm}
  May 2019 \\
  \vspace{4mm}
  Words: \\
  \end{center}
  }
}
\newpage

\begin{abstract}
  This is the thing working good \citep{turing}, here is another example \citep*{grayscott}
\end{abstract}

\newpage
\tableofcontents
\newpage 

\section{Introduction}

\section{Calculus}
Reaction diffusion relies on multivariable calculus, from the partial derivative to the laplacian
operator that governs diffusion. Multivariable calculus is a natural extension of calculus where instead of focusing on
the accumulation and change on 1 axis, that being y, it is possible to do in to any
number of dimensions. This means that if your function has more than 1 input, like for 3D functions, you would
have a new arsenal of tools and methods to analyse your equations. The most simple being the limit which
fundamentally operates the same as in the normal case. If there is a hole in the function but otherwise is
continuous, it would \say{fill in} that hole, and if there is a jump discontinuity the limit is defined some
certain directions but the limit itself is undefined. Once again using the concept of limits we can get the
integral and derivative as in normal calculus; however for this application we will be primarily focusing on
partial derivatives and other related operations. Instead derivatives in in multivariate partial derivatives
are used. These are computed in a similar form to derivatives but the input that is not differentiated is
treated as a constant. For use in the reaction equations the partial differential equations behave in a
similar way todifferential equations but rely on more inputs but are still the rate of change in a given input
over a function. 

\begin{gather}
 \frac{\partial f}{\partial x} = \partial_x{f} = \lim_{h \to 0} \frac{f(x+h, y) − f(x,y)}{h} \\
 \frac{\partial f}{\partial y} = \partial_y{f} = \lim_{h \to 0} \frac{f(x, y + h) − f(z,y)}{h}
\end{gather}

\begin{center}
 Example problem
 \begin{gather*}
  f(x, y) = xy^2 + x \\
  \partial_x f(x,y) = y^2 + 1 \\
  \partial_y f(x,y) = 2xy
 \end{gather*}
\end{center}

Then there are directional derivatives which are similar to the partial derivatives, but instead of being in a
component direction like along the x  or y direction it would be in the direction of the vector $\vec{u}$ with
components $ <a, b> $. These derivates can be done in any number of dimensions (2) but for reaction diffusion
formulas shown 2D cases (1) matter more.
\begin{gather}
  D_u f = \lim_{h \to 0} \frac{f(x + ha, y+ hb) − f(x, y)}{h}  \\
  D_u{f(\vec{x})} = \lim_{h \to 0} \frac{f(\vec{x} + hu) − f(\vec{x})}{h}
\end{gather}
The directional derivative can also be written as a dot products
\begin{gather*}
\begin{aligned}
  D_u f(x,y) &= a\partial_x f + b\partial_y f \\
             &= <\partial_x f, \partial_y f> * <a, b> \\
             &= <\partial_x f, \partial_y f> * u \\
             &= \nabla f(x,y) * u 
\end{aligned}
\end{gather*}

This $\nabla$, or nabla, is the sign for the grad operator read \say{del f} This represents a vector of all
of the partial derivatives that are possible in a function called the gradient. For example in $g(x, y, z)$,
grad g= the partial derivative of g in respect to x, y, and z. This gradient shoulds the direction of the
fastest increase of the function. So if we had a map of a mountain range the gradient of the map from a
point would be the fastest way to get to the top of the nearest mountain but not necessarily the highest
mountain. The gradient is often used for optimization problems since it can go \say{the highest} or most
optimal local  point in a function of any number of variables. A common abuse of notation is to set $\nabla$
to a value (3)
\begin{gather}
  \nabla = <\frac{\partial}{\partial x}, \frac{\partial}{\partial y}, \frac{\partial}{\partial z}>
\end{gather}

So the notation for the gradient makes since because you would be treating the function f as a scalar and
multiplying it to the $\nabla$ vector. This abuse of notation since it would also make sense for the
divergence of a vector field. For a vector field the function F (4) would be a vector function where you could
dot product of F and $\nabla$ (5) 
\begin{gather}
  \bold{F} = P\hat{\imath} + Q\hat{\jmath} + R\hat{k} \\
  div \bold{F} = \nabla * \bold{F} = \frac{\partial P}{\partial x} + \frac{\partial Q}{\partial y} + \frac{\partial R}{\partial z} 
\end{gather}

One can picture a vector field as being a model of fluid flow, where the vector at a point would be the
velocity of the fluid. Given this analogy, the divergence if the fluid is incompressible then the amount of
fluid that emanates from a point. In a vector field that would represented by all of the arrows pointing away
from a point and the arrows get larger around a point then the divergence $>$ 0 and if the arrows are smaller
or point inwards than the divergence $<$ 0. Both of these tools are useful but when applied together become
what is needed for the diffusion part of reaction diffusion that being the laplacian. The laplacian is the
divergence of the gradient of a function.

\begin{center}
Laplacian = $\nabla * (\nabla f) = \nabla^2 f = \frac{\partial^2 f}{\partial x^2} + \frac{\partial^2
f}{\partial y^2} \frac{\partial^2 f}{\partial z^2} = \sum_{i=1}^{n} \frac{\partial^2 f}{\partial x_i^2}$ \quad
$\nabla^2 = \nabla * \nabla$
\end{center}

This is called the Laplace operator or the Laplacian of a field because of its relation to Laplace's
Equation which has similarities to the heat transfer equation for diffusion. The reason that when put in
PDEs they exhibit this diffusion or spreading properly is due to their relation to the second derivative in
normal calculus. They identify local maxima and minima but in normal calculus where the second derivative is
0 in both cases the laplacian is highly positive in minima and highly negative in maxima. In the case of
PDEs, that would mean that the values would \say{move away} from maxima and \say{fill in} minima which exactly
describes osmosis and other diffusion systems.

\begin{center}
Laplace's Equation: $\nabla^2 \bold{F} = \frac{\partial^2 f}{\partial x^2} + \frac{\partial^2 f}{\partial
y^2} + \frac{\partial^2 f}{\partial z^2} = 0$
\end{center}

Equations that satisfies Laplace's Equation are called harmonic functions where at all points the value of
the laplacian = 0 

\section{Diffusion}
\section{Reaction}
\section{Reaction Diffusion}
\section{References}
\bibliographystyle{apa}
\bibliography{EE}
\newpage
\section{Appedix}
\lstinputlisting[language=Python, caption=Diffusion]{code/diffusion.py}
\lstinputlisting[language=Python, caption={Lotka-Volterra}]{code/lotkaVolterra.py}
\lstinputlisting[language=Python, caption=SIR]{code/SIR.py}
\end{document}
