\documentclass[12pt, letterpaper]{article}
\usepackage{color}
\usepackage{natbib}
\usepackage{parskip}
\usepackage{amsmath}
\usepackage{amssymb}
\usepackage{graphicx}
\usepackage{listings}
\usepackage{setspace}
\usepackage{geometry}
\usepackage{dirtytalk}
\usepackage{subcaption}
\usepackage{indentfirst}
\usepackage{anyfontsize}
\usepackage[utf8]{inputenc}

\parindent=0.5in

\graphicspath{{./imgs/}}

\newcommand{\sorta}[1]{\lq #1\rq \,}

\definecolor{codegray}{rgb}{0.2,0.2,0.2}
\definecolor{codepurple}{rgb}{0.58,0,0.82}
\definecolor{backcolor}{rgb}{0.95,0.95,0.92}

\lstdefinestyle{scheme}
  {backgroundcolor=\color{backcolor},
  commentstyle=\color{blue},
  keywordstyle=\color{magenta},
  numberstyle=\tiny\color{codegray},
  stringstyle=\color{codepurple},
  basicstyle=\footnotesize\ttfamily,
  morekeywords={*},
  breakatwhitespace=false,
  breaklines=true,
  captionpos=t,
  keepspaces=true,
  numbers=left,
  numbersep=5pt,
  showspaces=false,
  showstringspaces=false,
  showtabs=false,
  tabsize=2,
  title=\lstname,
  language=Python
}

\lstset{style=scheme}

\doublespace{}
\title{Discussing the Use of Reaction Diffusion Equations for the Simulation of Physical Phenomena}
\author{Simon Abrelat}
\date{\vspace{-5ex}}

\DeclareUnicodeCharacter{2212}{-}
\begin{document}

\large
{\fontsize{12}{14.4}
  {\singlespace{}
  \pagenumbering{gobble}
  \maketitle
  \begin{center}
  \vspace{4mm}
  002129--0004 \\
  \vspace{4mm}
  IB Extended Essay \\
  \vspace{4mm}
  February 2019 \\
  \vspace{4mm}
  Words: 2550 \\
  \end{center}
  }
}
\newpage

\pagenumbering{arabic}
\newpage
\tableofcontents
\newpage 

\section{Introduction} \label{introduction}
Reaction Diffusion equations are a class of equations with a wide range of uses and applications. They allow
for the modeling of the iterations between objects throughout space while passing through time. In other
words, they track the relationship between things over time as they spread and affect other things. This
is not a in-depth definition, but it serves to understand their purpose. It should be evident from 
the generality of their purpose that their applications are equally universal. As the name implies, they have
two sections: the reaction and the diffusion. The reaction serves as the
relation between objects and the diffusion is the spacial component. This is a discussion to visually
and intuitively explain the diffusion and reaction components and to tie them together to show the
complex and emergent behavior of Reaction Diffusion systems.


The thing that the Reaction Diffusion takes and manipulates or the \sorta{object} previously referred to is a
concentration variable, and the goal is to morph a concentration space. The equation takes a field of values and
returns a field or matrix of modified values. One way to think about the Reaction Diffusion is that there is
a reaction that spreads getting more \sorta{fuel} to continue the reaction. Another examples of this sort of reaction is a
disease that starts with one person and spreads based off of who touched it and stops when everyone has
recovered or died. Given the parameters of the equation there are a variety of intuitive explanations with
different levels of accuracy or mathematical formality. 

\section{Calculus} \label{calculus}
Reaction Diffusion relies on multivariable calculus, from the partial derivative to the Laplacian 
operator that governs diffusion. Multivariable calculus is a natural extension of calculus where instead of 
focusing on the accumulation or change on 1 axis, that normally being $y$, it is possible to run the calculus
operations for any
number of dimensions. This means that if a function has more than 1 input, for instance 3D functions,
there is a whole new arsenal of tools and methods to analyze these equations. The limit
fundamentally operates the same as in the 2 dimensional case. If there were a hole in the function it
would \sorta{fill in} that hole, and if there is a jump discontinuity the limit is
defined for certain directions as it approaches the discontinuity but the limit itself is undefined. Once
again, integral and derivative can be derived from the limits as in normal calculus; however, this
application will be primarily focusing on partial derivatives and other related operations. Instead
derivatives in multivariate partial derivatives are used. These are computed in a similar form to
derivatives but the inputs that are not
differentiated is treated as a constant. For use in the reaction equations the partial differential equations
behave in a similar way to differential equations except they have more inputs, but still give the rate 
of change for a given input over a function.

\begin{singlespace}
  \begin{gather}
    \frac{\partial f}{\partial x} = \partial_x{f} = \lim_{h \to 0} \frac{f(x+h, y) − f(x,y)}{h}
    \label{eq:partX}
  \end{gather}
  \begin{gather}
    \frac{\partial f}{\partial y} = \partial_y{f} = \lim_{h \to 0} \frac{f(x, y + h) − f(x,y)}{h}
    \label{eq:partY}
  \end{gather}

  \begin{center}
   Example problem
   \begin{gather*}
    f(x, y) = xy^2 + x \\
    \partial_x f(x,y) = y^2 + 1 \\
    \partial_y f(x,y) = 2xy
   \end{gather*}
  \end{center}
\end{singlespace}

\begin{figure}[t]
  \centering
  \begin{subfigure}[b]{.6\linewidth}
    \includegraphics[width=\linewidth]{Basics/multifn}
    \caption{$f(x,y) = xy^2 + x$}
  \end{subfigure}

  \begin{subfigure}[b]{.4\linewidth}
    \includegraphics[width=\linewidth]{Basics/partDervX}
    \caption{$\partial_x f(x,y) = y^2 + 1$}
  \end{subfigure}
  \begin{subfigure}[b]{.4\linewidth}
    \includegraphics[width=\linewidth]{Basics/partDervY}
    \caption{$\partial_y f(x,y) = 2xy$}
  \end{subfigure}
  \caption{Example functions}
\end{figure}

Directional derivatives are similar to the partial derivatives, but instead of being in 
a component direction, along the $x$ or $y$ direction, it would be in the direction of an arbitrary vector
,$\vec{u}$, with components $ <a, b> $. These derivatives can be done in any number of dimensions 
(\ref{eq:NDirectional}) but for Reaction Diffusion formulas shown 2D cases (\ref{eq:2Directional}) matter
more.

\begin{equation}
  D_u{f(\vec{x})} = \lim_{h \to 0} \frac{f(\vec{x} + hu) − f(\vec{x})}{h}
  \label{eq:NDirectional}
\end{equation}
\begin{equation}
  D_u f = \lim_{h \to 0} \frac{f(x + ha, y+ hb) − f(x, y)}{h}
  \label{eq:2Directional}
\end{equation}
The directional derivative can also be written as a dot products.
\begin{gather*}
\begin{aligned}
  D_u f(x,y) &= a\partial_x f + b\partial_y f \\
             &= <\partial_x f, \partial_y f> \cdot <a, b> \\
             &= <\partial_x f, \partial_y f> \cdot u \\
             &= \nabla f(x,y) * u 
\end{aligned}
\end{gather*}

This $\nabla$, or nabla, is the sign for the grad operator. This represents a vector of all
of the partial derivatives for all of the function inputs, which is called the gradient. For example in $g(x,
y, z)$, the gradient of $g$ is the vector where each component is the partial derivative of $g$ for its
inputs $x$, $y$, and $z$. This gradient points in the direction of the \sorta{fastest increase} of the
function. Picture a map of a mountain range the gradient of the map from a point would be the fastest
way to get to the top of the nearest mountain but not necessarily the highest mountain. The gradient is often
used for optimization problems since it can go \sorta{the highest} or local maxima in a function of any
number of variables. A common abuse of notation is to set $\nabla$ to a value (\ref{eq:nabla})

\begin{equation}
  \begin{aligned}
    \nabla &= <\frac{\partial}{\partial x}, \frac{\partial}{\partial y}, \frac{\partial}{\partial z}> \\
           &= \sum_{i=1}^{n} \vec{e}_{i} \: \frac{\partial}{\partial x_i}
  \end{aligned}
  \label{eq:nabla}
\end{equation}

\begin{figure}[!h]
  \centering
  \begin{subfigure}[b]{.45\linewidth}
    \includegraphics[width=\linewidth]{Basics/multifn}
    \caption{$f(x,y) = xy^2 + x$}
  \end{subfigure}
  \begin{subfigure}[b]{.5\linewidth}
    \includegraphics[width=\linewidth]{Basics/gradient}
    \caption{the gradient vector field: $\nabla * f(x,y)$}
  \end{subfigure}
\end{figure}

The notation for the gradient makes sense the function, $f$, would be treated as a scalar and
multiplying it by the $\nabla$ vector. This abuse of notation since it would also make sense for the
divergence of a vector field. For a vector field the function $\mathbf{F}$ (\ref{eq:vectorFn}) would be 
a vector function of the dot product of $\mathbf{F}$ and $\nabla$ (\ref{eq:div})

\begin{equation}
  \mathbf{F} = P\hat{\imath} + Q\hat{\jmath} + R\hat{k} \\
  \label{eq:vectorFn}
\end{equation}
\begin{equation}
  div \, \mathbf{F} = \nabla * \mathbf{F} = \frac{\partial P}{\partial x} + \frac{\partial Q}{\partial y} +
 \frac{\partial R}{\partial z}
  \label{eq:div}
\end{equation}

\begin{figure}[!h]
  \centering  
  \label{fig:div}
  \includegraphics[width=.4\linewidth]{Basics/divergence}
  \caption{the gradient vector field: $\nabla * f(x,y)$}
\end{figure}

\newpage

The divergence being the dot product of the partial derivative vector makes sense since it is the sum of the
derivatives or \sorta{velocities}.
One can picture a vector field as being a model of fluid flow, where the vector at a point is 
velocity of the fluid. Given this analogy, the divergence is the rate that the fluid \sorta{speeds up} at a
point. The divergence if positive when fluid \sorta{speeds up} and negative if it \sorta{slows down}. In a
vector field represented by all of the arrows pointing away from the origin and the arrows get larger around
a point when the divergence $>$ 0 and if the arrows are smaller or point to the origin when the divergence
$<$ 0. Both of these tools are useful but when applied together become what is needed for the Diffusion part
of Reaction Diffusion, that being the laplacian. The laplacian (\ref{eq:laplacian}) is the divergence of the
gradient of a function.
\begin{equation}
  \label{eq:laplacian}
  \begin{split}
\nabla^2 f &= \nabla * (\nabla f) \\
           &= \frac{\partial^2 f}{\partial x^2} + \frac{\partial^2 f}{\partial y^2} + \frac{\partial^2
             f}{\partial z^2} \\
           &= \sum_{i=1}^{n} \frac{\partial^2 f}{\partial x_i^2}
\end{split}
\end{equation}

Equation~\ref{eq:laplacian} is called the Laplace operator, or the Laplacian of a field. The reason that,when
put in partial differiental equations
(PDEs), they exhibit this diffusion or spreading property, is due to their relation to the second derivative in
normal calculus. They identify local maxima and minima, but in normal calculus where the second derivative is
$0$, in both cases the Laplacian is highly positive in minima and highly negative in maxima. 
The Laplacian is the sum of second partial derivatives for all of the function's inputs. In the case 
of PDEs, the laplacian \sorta{fills in} the minima and \sorta{pushes down} the
maxima. This process is what drives osmosis and heat diffusion.

\section{Diffusion} \label{diffusion}

Diffusion is the spread of things from high to low densities. A simple example would be diffusion with heat,or
osmosis with water. The mathematical way this is described is through a partial
differential equation using the Laplace operator, $ \nabla^2 $.

\begin{equation}
  \frac{\partial u}{\partial t} = \alpha \nabla^2 u
  \label{eq:heatDiff}
\end{equation}
\begin{equation}
  \frac{\partial u}{\partial t} = \alpha (\frac{\partial^2 u}{\partial x^2} + \frac{\partial^2 u}{\partial
  y^2} + \frac{\partial^2 u}{\partial z^2})
  \label{eq:heatDiff3}
\end{equation}

Equation~\ref{eq:heatDiff} is a simple partial differential equation that has $u(x, y, z)$ 
to determine the heat and $\alpha$ as the coefficient of thermal diffusivity. Because of the Laplace
operator, the equation does not need to show its inputs which also allows for a $n$-dimensional continuation.
In 3 dimensions, this equation can be expanded into equation~\ref{eq:heatDiff3}.

\begin{figure}[h]
  \caption{Heat Diffusion Progression}
  \label{fig:diffusionProgression}
  \centering
  \begin{subfigure}[b]{.21\linewidth}
    \includegraphics[width=\linewidth]{HeatProgression/diffusion0}
    \caption{$n=0$}
  \end{subfigure}
  \begin{subfigure}[b]{.21\linewidth}
    \includegraphics[width=\linewidth]{HeatProgression/diffusion1000}
    \caption{$n=1000$}
  \end{subfigure}
  \begin{subfigure}[b]{.21\linewidth}
    \includegraphics[width=\linewidth]{HeatProgression/diffusion5000}
    \caption{$n=5000$}
  \end{subfigure}
  \begin{subfigure}[b]{.21\linewidth}
    \includegraphics[width=\linewidth]{HeatProgression/diffusion8000}
    \caption{$n=8000$}
  \end{subfigure}
\end{figure}

Figure~\ref{fig:diffusionProgression} shows the diffusion from 2 point sources of heat that eventually
merge into a single blob that would continue to spread throughout the plane.

The heat diffusion equation (\ref{eq:heatDiff}) uses the Laplacian which \sorta{pushes} the local maxima 
down 
and the local minima up. Later in the Reaction Diffusion equations, the Laplacian is used to give the
equations properties based off their location and physical proximity to other elements. This diffusion
section gives reaction diffusion equations their interesting properties since previously converging
reactionary systems remain in this possibly unstable equilibrium. 

\section{Reaction} \label{reaction}
Reaction systems that typically contain multiple ordinary or partial differential 
equations. Where diffusion equations usually have only one variable, reaction systems are often 
multi-variable. There are many different examples of reaction systems, and this section is going to focus on
the logistic equation and SIR and Lotka-Volterra models.

\subsection{Logistic Equation} \label{logistic}

The most simple and generally recognized ordinary differential equation reaction system, especially relating
to population, would be the logistic equation (\ref{eq:logistic}) which is a $1$ component system that shows
the growth and maximum population of an area.

\begin{singlespace}
  \begin{equation}\label{eq:logistic}
    \frac{dP}{dt} = rP (1 - \frac{P}{K})
  \end{equation}
  \begin{small}
$r$: growth rate \\
$K$: carrying capacity
  \end{small}
\end{singlespace}

\begin{figure}[h]
  \caption{Logistic equation $p_0=3$ $k=100$}
  \label{fig:progression}
  \centering
  \begin{subfigure}[b]{.3\linewidth}
    \includegraphics[width=\linewidth]{Logistic/logistic01.png}
    \caption{$r=0.1$}
  \end{subfigure}
  \begin{subfigure}[b]{.3\linewidth}
    \includegraphics[width=\linewidth]{Logistic/logistic02.png}
    \caption{$r=0.2$}
  \end{subfigure}
  \begin{subfigure}[b]{.3\linewidth}
    \includegraphics[width=\linewidth]{Logistic/logistic03.png}
    \caption{$r=0.3$}
  \end{subfigure}
\end{figure}

The equation that describes the S-curve of growth and decay of the population. $rP$ is the
unencumbered growth rate of the population, which is proportional to the current population.
$1- \frac{P}{K}$ is the limiting factor for the logistic equation. As your population approaches the carrying
capacity the growth slows, so the point they are equal and there is no more growth. If the population
increases over the carrying capacity it also pulls the population back into equilibrium This equation was
originally generated for describing Malthusian population growth, and was later rediscovered a number of
times. An examples of the use of the logistic equation is in modeling of tumor growth \citep{logisticTumor}.

\subsection{Lotka-Volterra Model} \label{lotkavolterra}

Another, and more complex, reaction system is the Lotka-Volterra model (\ref{eq:lotkaVolterra}). This is used
to model the populations of predators and pray as they grow and shrink in a cyclical manner, making it a 2
component system, in the case $x$ for prey and $y$ for predators. This is one of the earliest Predator-Pray
models based on sound mathematical principles. There are some problems with such a simple model, but the
biggest issue is, in reality populations, reach an equilibrium, but this model is cyclic. There are
also cases were there is a cyclic growth off to infinity, becoming unstable \citep{lotkaVolterra}. These
issues mainly come from a lack of 
gender differences, size effects such as overpopulation and over consumption of resources, and a variety of
outside influences. Hence why this is only used to describe specific predator-prey examples such as hare and lynx populations. 

\begin{singlespace}
  \begin{equation}\label{eq:lotkaVolterra}
    \begin{split}
      &\frac{dx}{dt} = \alpha x - \beta x y \\
      &\frac{dy}{dt} = - \gamma y + \delta x y
    \end{split}
  \end{equation}
  \begin{small}
$\alpha$: the intrinsic rate of pray population growth \\
$\beta$: predation rate coefficient \\
$\gamma$: the predator mortality rate. \\
$\delta$: reproductive rate of the predators per pray eaten \\
  \end{small}
\end{singlespace}

\begin{figure}[!h]
  \caption{Examples of the Lotka-Volterra Model}
  \label{fig:lotkaVolterra}
  \begin{center}
    \begin{subfigure}[b]{.45\linewidth}
      \includegraphics[width=\linewidth]{LotkaVolterra/lotkaVolterra1}
      \caption{$ \alpha = \beta = \gamma = \delta = 1$}
    \end{subfigure}
    \begin{subfigure}[b]{.45\linewidth}
      \includegraphics[width=\linewidth]{LotkaVolterra/lotkaVolterra2}
      \caption{$ \alpha = \frac{2}{3}, \beta = \frac{4}{3}, \gamma = 1, \delta = 1 $}
    \end{subfigure}
  \end{center}
\end{figure}


The benefit of such a simple model is the simple explanation. For prey, $\alpha x$ is clearly the unencumbered
growth rate and $-\beta x y$ is the rate of predation that is influenced by the size of both the predator and
prey populations. The predators explanation is a little more complicated. For predators, the $\delta x y$ is
the increase in predator population, given that each predator reproduces at a rate $\delta$ per prey eaten.
The $- \gamma y$ is the mortality rate for the predators. This relates to Reaction Diffusion since they
illustrates how multiple systems can be used in relation to each other before Diffusion.

\subsection{SIR Model} \label{sir}

The SIR model (\ref{eq:SIR}), or Susceptible-Infected-Recovered Model, is a computational epidemiology and
serves as the core of other more advanced models, despite already having 3 components. This model is
fairly intuitive. The susceptible equation is similar to the predation in the Lotka-Volterra Model, were the
population decreases by the rate of spread times the infected and the susceptible, $- \beta s i$. 
Therefore, if the size of the infected or susceptible populations are low, then there are few new cases of 
illness. The recovering is simply decreasing by the recovery rate times infected, $\kappa r$.  The
infected change is equal to the newly infected minus the recovered, $\beta s i - \kappa r$. 
Figure~\ref{fig:SIR}a is an example for a typical illness that is not an epidemic. In comparison, Figure~\ref{fig:SIR}b is an example of an epidemic.

\begin{singlespace}
  \begin{equation}\label{eq:SIR}
    \begin{split}
      &\frac{dS}{dt} = -\beta s i \\
      &\frac{dI}{dt} = \beta s i - \kappa r \\
      &\frac{dR}{dt} = \kappa r
    \end{split}
  \end{equation}
  \begin{small}
$\kappa$: the number of days until recovery \\
$\beta$: the number of days needed to be in contact to spread
  \end{small}
\end{singlespace}

\begin{figure}[!h]
  \caption{Examples of SIR model}
  \label{fig:SIR}
  \begin{center}
    \begin{subfigure}[b]{.35\linewidth}
      \includegraphics[width=\linewidth]{SIR/SIR_Start}
      \caption{$\kappa = \frac{1}{3}, \beta = \frac{1}{2}$}
    \end{subfigure}
    \begin{subfigure}[b]{.35\linewidth}
      \includegraphics[width=\linewidth]{SIR/SIR_Epi}
      \caption{$\kappa = \frac{1}{6}, \beta =  1$}
    \end{subfigure}
  \end{center}
\end{figure}

\section{Reaction Diffusion} \label{reactiondiffusion}

Reaction Diffusion is the merger of diffusion and reaction systems with emergent properties. The
generalized formula (\ref{eq:genRD}) works in any number of dimensions and with any number of components. To
limit algorithmic complexity, only the 1\textsuperscript{st} and 2\textsuperscript{nd} dimensional cases will be discussed. $R(\vec{q})$ is a vector
equation that represents the reaction systems described in Section~\ref{reaction}. To complete the name, the 
$\underline{\underline{\mathbf{D}}} \nabla^2 \vec{q}$ then represents the diffusion for any number of
dimensions, exactly like heat diffusion described in Section~\ref{diffusion}.

\begin{singlespace}
  \begin{equation}
    \label{eq:genRD}
    \frac{\partial \vec{q}}{\partial t} = − \underline{\underline{\mathbf{D}}} \nabla^2 \vec{q} + R(\vec{q})
  \end{equation}
  \begin{small}
    $\underline{\underline{\mathbf{D}}}$ is a diagonalized weight matrix \\
    $\vec{q}$: a vector of inputs given a time \\
    $R(\vec{q})$: a function that takes into account local reactions
  \end{small}
\end{singlespace}

\subsection{Fisher's equation} \label{fisher}
The first reaction diffusion system is Fisher's equation (\ref{eq:fisher}) which is a 1 component, 1
dimensional system that represents the wave front switching between equilibrium states, such as a switch of
dominant genetic traits \citep{fisher}.

\begin{equation}
  \label{eq:fisher}
  \frac{\partial u}{\partial t} - D \frac{\partial^2 u}{\partial x^2} = ur(1 - u) 
\end{equation}

\begin{figure}[h]
  \caption{Fisher Equation Example}
  \label{fig:fisherGraphs}
  \centering
  \begin{subfigure}[b]{.23\linewidth}
    \includegraphics[width=\linewidth]{Fisher/f0.png}
    \caption{$t=0$}
  \end{subfigure}
  \begin{subfigure}[b]{.23\linewidth}
    \includegraphics[width=\linewidth]{Fisher/f50.png}
    \caption{$t=50$}
  \end{subfigure}
  \begin{subfigure}[b]{.23\linewidth}
    \includegraphics[width=\linewidth]{Fisher/f500.png}
    \caption{$t=500$}
  \end{subfigure}
  \begin{subfigure}[b]{.23\linewidth}
    \includegraphics[width=\linewidth]{Fisher/f10000.png}
    \caption{$t=10000$}
  \end{subfigure}
\end{figure}

The Fisher equation (\ref{eq:fisher}) is related to the logistic equation (\ref{eq:logistic}) previously
described. The logistic equation can be thought of as the $R(\vec{q})$ given in the generalized
reaction diffusion equation (\ref{eq:genRD}) and the $\dfrac{\partial^2 u}{\partial t^2}$ is the 
$ - \underline{\underline{\mathbf{D}}} \nabla^2 \vec{q} $ of the general form. The Fisher equation is not as
generalized as the logistic equation, since the \sorta{carrying capacity} is 1. These equations are
still useful for illustrating the relation between reaction and reaction-diffusion systems.

\subsection{Cahn-Hilliard} \label{cahnhilliard}

The Cahn-Hilliard equation (\ref{eq:cahnHilliard}) is a 1 component, $n$ dimensional system related to phase 
separation \citep{cahnhilliard}. Where Fisher's equation (\ref{eq:fisher}) is about the phase transition in 1
dimension, which is a wave,
the Cahn-Hilliard equation can be used to describe the phase transition on a plane or any space,
$\mathbb{R}^n$. This system can be applied to a multitude of uses, a theme of reaction diffusion equations.

\begin{singlespace}
  \begin{equation}\label{eq:cahnHilliard}
    \begin{split}
      \frac{\partial c}{\partial t} &= D \nabla^2 (c^3 -c -\gamma \nabla^2 c) \\
                                    &= D \nabla^2 \mu 
    \end{split}
  \end{equation}
  \begin{small}
$c$: the concentration of the liquid \\
$D$: the diffusion coefficient \\
$\sqrt{\gamma}$: the length of the transition regions between the domain \\
$\mu = (c^3 - c -\gamma \nabla^2 c)$: the called the chemical potential 
  \end{small}
\end{singlespace}

\begin{figure}[h]
  \caption{Cahn-Hilliard Examples at $t=2000$}
  \label{fig:cahnHilliardGraphs}
  \centering
  \begin{subfigure}[b]{.23\linewidth}
    \includegraphics[width=\linewidth]{CahnHilliard/ch0.png}
    \caption{Initial Seed}
  \end{subfigure}
  \begin{subfigure}[b]{.23\linewidth}
    \includegraphics[width=\linewidth]{CahnHilliard/ch125.png}
    \caption{$\gamma=0.125$}
  \end{subfigure}
  \begin{subfigure}[b]{.23\linewidth}
    \includegraphics[width=\linewidth]{CahnHilliard/ch375.png}
    \caption{$\gamma=0.375$}
  \end{subfigure}
  \begin{subfigure}[b]{.23\linewidth}
    \includegraphics[width=\linewidth]{CahnHilliard/ch5.png}
    \caption{$\gamma=0.5$}
  \end{subfigure}
\end{figure}

One such use is in high-strength alloys \citep{challoys}. An intuitive way of thinking about this process,
despite losing some of the generality, is thinking about the Cahn-Hilliard equation as the interaction
between insoluble fluids. If the $\gamma$, chemical potential, is high enough, as $t$ approaches
$\infty$, the two insoluble fluids completely separate. It is clear that the Cahn-Hilliard equation is very
useful for studying emulsions and solutions as shown by \cite{CHEmulsions} and other studies.  

\subsection{Gray-Scott}\label{grayscott}

The Gray-Scott Model is the reaction diffusion equation for computational art. Originally, it was for modeling
a specific chemical reaction \citep{grayscott}. Due to its generality and interesting patterns it
became used for making natural-looking patterns with a seed, for example the Gray-Scott lamp \citep{reactionLamp}. 
The original reaction takes in $a$ and $b$ then produces a byproduct, $p$. $k$ is the rate that $b$ turns
into the nonreactive $p$. $f$ is the rate that $a$ is added and the other components are removed.
The Gray-Scott model is also used computational artists. Its main uses are for modeling patterns in biology.

\begin{singlespace}
  \begin{equation}\label{eq:grayscott}
    \begin{split}
      &\frac{\partial a}{\partial t} = d_a\nabla^2a − ab^2 + f(1−a) \\
      &\frac{\partial b}{\partial t} = d_b\nabla^2b + ab^2 − (k + f)b
    \end{split}
  \end{equation}
  \begin{equation}
    \begin{split}
      a + 3b &\rightarrow 3b \\
           b &\rightarrow p
    \end{split}
  \end{equation}
  \begin{small}
$d_a$: the rate of growth for a \\
$d_b$: the rate of growth for b \\
$f$: the feed rate also keeps a from exceeding 1 \\
$k$: the kill rate of b which is scaled by b to prevent it from being less than 0
  \end{small}
\end{singlespace}

\begin{figure}[h]
  \caption{Coral Simulation}
  \label{fig:coralProgression}
  \centering
  \begin{subfigure}[b]{.22\linewidth}
    \includegraphics[width=\linewidth]{GrayScott/coral0.png}
    \caption{$r=0.1$}
  \end{subfigure}
  \begin{subfigure}[b]{.22\linewidth}
    \includegraphics[width=\linewidth]{GrayScott/coral1000.png}
    \caption{$r=0.2$}
  \end{subfigure}
  \begin{subfigure}[b]{.22\linewidth}
    \includegraphics[width=\linewidth]{GrayScott/coral5000.png}
    \caption{$r=0.3$}
  \end{subfigure}
  \begin{subfigure}[b]{.22\linewidth}
    \includegraphics[width=\linewidth]{GrayScott/coral10000.png}
    \caption{$r=0.3$}
  \end{subfigure}
\end{figure}

These patterns, also known as Turing patterns, can be used to generate anything from mitosis and spirals to zebrafish patterns. An example of the
differences between Reaction and Reaction Diffusion systems is, firstly, the spatial component, and, secondly, how
the patterns change given their environment. In animal Turing patterns, the area and type of skin effects the
pigment patterns. For leopards, the patterns of the patches are not homogeneous. Around the center of their
sides, the patches are normally larger and decrease in size as they go to the paws or face 
\citep{turingsCat}.

\section{Conclusion}
Generality is both a blessing and a curse. Reaction Diffusion systems are incredibly broad.
Trying to understand \sorta{movement} and \sorta{flow} in higher dimensions between multiple components
is confusing, but once understood, their applications are endless. Calculus is at the core of Reaction
Diffusion. Most of the equations are partial differential equations, and
all of Reaction Diffusion contains the Laplace operator. Reaction equations are what govern the interactions
between objects, which change over time using differential equations. Diffusion governs how reactions
change over space in Reaction Diffusion systems. Emulsifications, disease, populations,
patterns, and even things such as the probably of particles in quantum systems can be modeled using reaction
diffusion equations. The problem is, with such generality, understanding how and, more
importantly, why these equations work is difficult.

\newpage
\bibliographystyle{apa}
\bibliography{EE}
\newpage
\section{Appendix}

\listoffigures

\lstinputlisting[language=Python, label={lst:surfaceCode}, caption={Surface}]{code/surface.py}
\lstinputlisting[language=Python, label={lst:gradientCode}, caption={Gradient}]{code/gradient.py}
\lstinputlisting[language=Python, label={lst:divergenceCode}, caption={Divergence}]{code/divergence.py}
\lstinputlisting[language=Python, label={lst:diffusionCode}, caption={Diffusion}]{code/diffusion.py}
\lstinputlisting[language=Python, label={lst:logisticCode}, caption={Logistic Equation}]{code/logistic.py}
\lstinputlisting[language=Python, label={lst:lotkaVolterraCode}, caption={Lotka-Volterra}]{code/lotkaVolterra.py}
\lstinputlisting[language=Python, label={lst:SIRCode}, caption={SIR}]{code/SIR.py}
\lstinputlisting[language=Python, label={lst:cahnHilliardCode}, caption={Cahn-Hilliard}]{code/cahnHilliard.py}
\lstinputlisting[language=Python, label={lst:grayScottCode}, caption={Gray-Scott}]{code/grayScott.py}
\end{document}
