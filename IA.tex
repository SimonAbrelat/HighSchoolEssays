\documentclass[12pt, letterpaper]{article}
\usepackage{parskip}
\usepackage{graphicx}
\usepackage{geometry}
\usepackage{setspace}
\usepackage{subcaption}
\usepackage{anyfontsize}
\usepackage{indentfirst}
\usepackage[utf8]{inputenc}
\usepackage[american]{babel}
\usepackage[babel]{csquotes}
\usepackage[authordate,isbn=false,backend=biber]{biblatex-chicago}

\addbibresource{IA.bib}
\defbibheading{bibliography}{\section{Bibliography}}
\bibliography{IA}
\renewbibmacro*{cite:ibid}{\printtext[bibhyperref]{\bibstring[\mkibid]{ibidem}}}

\geometry{letterpaper, portrait, margin=1in}
\graphicspath{{../imgs/}}

\title{Media and its Role in the Development and Downfall of \Aum{}}
\author{Simon Abrelat}
\date{May 2019}

\newcommand{\sorta}[1]{`#1'}
\newcommand{\poses}[1]{#1's}
\newcommand{\say}[1]{``#1''}
\newcommand{\Aum}[0]{Aum Shinriky\=o }

\begin{document}
%\large
\doublespace{}
\parindent=0.5in

{\fontsize{12}{14.4}
  {\singlespace
    \pagenumbering{gobble}
    \maketitle
    \begin{center}
    002129-0004 \\
    \vspace{4mm}
    IB History HL IA \\
    \vspace{4mm}
    Words:  \\ % words
    \end{center}
  }
}	


\newpage
\tableofcontents
\pagenumbering{arabic}
\newpage

\section{Evaluation of Sources}
\Aum{} is a doomsday cult that was based in Japan, which is primarily known for their Sarin gas
attacks on the Tokyo subway system. It thrived on the spiritual void in the \poses{80} and  \poses{90},
and used popular culture for their recruitment. A cult lives and breathes on its members and perception,
just think about the siege of the Branch Divisions at Waco, Texas, the mass murder suicides of the Order of
the Solar Temple, and Jonestown. Cults get violent when there practices get brought to the light. So, to
what extent can the popularity and the downfall of \Aum{} be attributed to the media?

\subsection{\citetitle{MediaStages}}
% discuss V and L thru C and O and P
% OPcVL
\say{Reactions to the Aum Affair: The Rise of the \sorta{Anti-Cult} Movement in Japan.} is an essay is
based on a paper presented at the annual meeting of American Academy of Religion in 1996
\footcite{MediaStages}. It was published in the 21\textsuperscript{st} .Volume of the annual Bulletin for
the Nanzan Institute for Religion and Culture. This separates the \say{Aum Affair}, the time between being
approved as a religious institution and the gas attacks, into four different periods.The first period was
as it budded into its current form, the second is the beginning of negative press, the third is a time were
\say{intellectuals reevaluated} \Aum, and finally, the forth is the gas attacks to there downfall \footcite[33]{MediaStages}. This
resource is valuable because of how it tracks the criticism of the cult and they way \Aum responded. Their
typical response being murder. This paper showed the causal relationship between the level of criticism and
violence which led to their downfall. The main issue with this paper is that it does not explain the media
practices used by \Aum, the cultural context, or what made them gain so much power. The main purpose of 
this paper was to show how new religious or spiritual movements can clash with culture at large for an
academic conference.

\subsection{\citetitle{OtakuWired}}

This source is a WIRED article in 1996. This was one year after the infamous gas attack. The main purpose 
the article is to illustrate the sort of people that would get sucked into this cult, the Otaku. An otaku
is a nerd that immerses themself in anime, manga, and Japanese cartoons and popular culture to the point of
becoming detached with reality and in generally a derogatory term. The stereotype is also focused around 
nerdy men. The article reflects the mainstream perception of the cult members, that being lost, young, 
otaku men. It gives a few second hand accounts from former cult members and highlights examples of 
prominent scientists that got taken in. The limitations are that there are no examples of their recruitment
materials that cater to otaku, and how they continue with the stereotype of demeaning otaku. They mention
the societal issues, but attributed most of the recruitment to violent anime and science fiction stories.
This article also shows the attitude toward the cult at time when they were at their most well-known. 

\newpage
\printbibliography{}

\newpage
\section{Appendix}

\end{document}
