\documentclass[12pt, letterpaper]{article}
\usepackage{parskip}
\usepackage{graphicx}
\usepackage{geometry}
\usepackage{setspace}
\usepackage{subcaption}
\usepackage{anyfontsize}
\usepackage{indentfirst}
\usepackage[utf8]{inputenc}
\usepackage[american]{babel}
\usepackage[babel]{csquotes}
\usepackage[authordate,isbn=false,backend=biber]{biblatex-chicago}

\addbibresource{IA.bib}
\defbibheading{bibliography}{\section{Bibliography}}
\bibliography{IA}
\renewbibmacro*{cite:ibid}{\printtext[bibhyperref]{\bibstring[\mkibid]{ibidem}}}

\geometry{letterpaper, portrait, margin=1in}
\graphicspath{{../imgs/}}

\title{Media and its Role in the Development and Downfall of Aum Shinriky\=o}
\author{Simon Abrelat}
\date{May 2019}

\newcommand{\sorta}[1]{`#1'}
\newcommand{\poses}[1]{#1's}
\newcommand{\say}[1]{``#1''}
\newcommand{\Aum}[0]{Aum Shinriky\=o }

\begin{document}
%\large
\doublespace{}
\parindent=0.5in

{\fontsize{12}{14.4}
  {\singlespace
    \pagenumbering{gobble}
    \maketitle
    \begin{center}
    002129-0004 \\
    \vspace{4mm}
    IB History HL IA \\
    \vspace{4mm}
    Words:  \\ % words
    \end{center}
  }
}	


\newpage
\tableofcontents
\pagenumbering{arabic}
\newpage

\section{Evaluation of Sources}
Aum Shinriky\=o is a doomsday cult that was based in Japan, which is primarily known for their Sarin gas
attacks on the Tokyo subway system. It thrived on the spiritual void in the \poses{80} and  \poses{90},
and used popular culture for their recruitment. A cult lives and breathes on its members and perception,
just think about the siege of the Branch Divisions at Waco, Texas, the mass murder suicides of the Order of
the Solar Temple, and Jonestown. Cults get violent when there practices get brought to the light. So, to
what extent can the popularity and the downfall of Aum Shinriky\=o be attributed to the media?

\subsection{\citetitle{manabu_reactions_1997}}
\say{Reactions to the Aum Affair: The Rise of the \sorta{Anti-Cult} Movement in Japan.} is an essay is based
on a paper presented at the annual meeting of American Academy of Religion in 1996
\footcite{manabu_reactions_1997}. It
was published in the 21\textsuperscript{st} .Volume of the annual Bulletin for the Nanzan Institute for
Religion and Culture. This separates the \say{Aum Affair}, the time between being approved as a religious
institution and the gas attacks, into four different periods.Th  first period was as it budded into its
current form, the second is the beginning of negative press, the third is a time were \say{intellectuals 
\footcite[33]{manabu_reactions_1997} reevaluated} Aum Shinriky\=o, and finally, the forth is the gas attacks to there downfall. This resource is valuable because of how it tracks the criticism of the cult and
they way Aum Shinriky\=o responded. Their typical response being murder. This paper showed the causal
relationship between the level of criticism and violence which led to their downfall. The main issue with
this paper is that it does not explain the media practices used by Aum Shinriky\=o, the cultural context, or
what made them gain so much power. The main purpose of  this paper was to show how new religious or spiritual
movements can clash with culture at large for an academic conference.

\subsection{\citetitle{noauthor_cult_nodate}}

This source is a WIRED article in 1996 \footcite{noauthor_cult_nodate}. This was one year after the infamous gas attack. The main purpose 
the article is to illustrate the sort of people that would get sucked into this cult, the Otaku. An otaku
is a nerd that immerses themself in anime, manga, and Japanese cartoons and popular culture to the point of
becoming detached with reality and in generally a derogatory term. The stereotype is also focused around 
nerdy men. The article reflects the mainstream perception of the cult members, that being lost, young, 
otaku men. It gives a few second hand accounts from former cult members and highlights examples of 
prominent scientists that got taken in. The limitations are that there are no examples of their recruitment
materials that cater to otaku, and how they continue with the stereotype of demeaning otaku. They mention
the societal issues, but attributed most of the recruitment to violent anime and science fiction stories.
This article also shows the attitude toward the cult at time when they were at their most well-known. 

\section{Introduction To Aum}
Aum Shinriky\=o is a Japanese cult that started operating in 1987 and was granted status as an Religious
Cooperation in 1989 by the Tokyo Metropolitan Government. The cult was started by Asahara Shoko, birth-name
Chizuo Matsumoto, who is believed to subject to \textit{pseudologia phantastica}
\footcite[5]{olson_aum_1999}, otherwise known as pathological lying. He later developed sever
incarceration psychosis \footcite[6]{olson_aum_1999} over the course of his trials before getting
the death penalty. This psychiatric analysis supports how media and other religions effected \poses{Aum}
actions and beliefs. Aum is an outstretch of a larger \sorta{New Religion} wave in Japan, and was, in many
ways, an offshoot of Agonshu \footcite[82,83]{watanabe_religion_1998}. During his time with Agonshu, he
started to mix Buddhism and Yoga. Not Yoga like the contemporary activity of housewives, but one of the sixes
schools of Hinduism. With his connection to Yoga, he chose Shiva as his main deity. For his own Buddhist
practice, he went to Tantra Vajrayana which is a Tibetan esoteric Buddhism.

\subsection{Beliefs}
The beliefs of Aum are a complex mix of Buddhism, Yoga, and aspects of Christianity, more specifically the
idea of a messianic leader \footcite[1144]{metraux_religious_1995}. For Buddhism, the Vajrayana tradition is 
a
subcategory of the Mahayana sutras and the borrowed the idea of Ten Realms (jikkai) from the Lotus sutra, a
different sub-division of the Mahayana branch of Buddhism. When someone is reincarnated based on their karma,
or level of morality as determined by their actions and thoughts), they could return to any of the six rebirth realms: 
Hell, Hungry Ghosts (pretas), Beasts, Titans (Asuras), and Humans \footcite[83]{watanabe_religion_1998}. Aum also uses the idea of the Three Ages 
of Buddhism: sh\=ob\=o, z\=oh\=o, and mapp\=o. Sh\=ob\=o is the period were Dharma, or behavior in accordance
to the cosmic law, is at its highest, and Mapp\=o, the Degenerate age, is were Dharma has all but 
disappeared. Asahara claimed that he had awoken his Kundalini, serpent power, which is located in the lower
spine. It is seen as the origin of supernatural and the energy source for spiritual. Later, he co-opted the
idea of Poa, or phowa, from more violent esoteric Buddhist sects. Poa is a Tibetan term for the transference
of consciousness, or, in other words, death. With a skilled enough practitioner could transfer a 
consciousness to a higher realm. Asahara weaponized this concept to make killing at the order or a spiritual
leader, him, virtuous. They believed that the world was in such a state of moral decline that a beast is more
likely to reincarnate to a higher realm than a human. Asahara he had a revelation through Shiva and has
reached the tenth realm of buddhahood. Shiva had told him that he would bring in a age of sh\=ob\=o, and
usher in Shambhala. Now Poa also serves the purpose of bringing the cleansing of Mapp\=o, raising the Dharma
of the new kingdom and raising a human from the realm of hell. The goal of Aum was to cause the Harumagedon 
or Armageddon to cause the second period of Sh\=ob\=o.

\newpage
\printbibliography{}

\newpage
\section*{Appendix}

\end{document}
